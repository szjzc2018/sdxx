\documentclass[a4 paper,12pt]{article}
\usepackage[inner=2.0cm,outer=2.0cm,top=2.0cm,bottom=2.0cm]{geometry}
\linespread{1.1}
\usepackage{setspace}
\usepackage[rgb]{xcolor}
\usepackage{verbatim}
\usepackage{subcaption}
\usepackage{fancyhdr}
\usepackage{fullpage}
\usepackage[colorlinks=true, urlcolor=blue, linkcolor=blue, citecolor=blue]{hyperref}
\usepackage{booktabs}
\usepackage{amsmath,amsfonts,amsthm,amssymb}
\usepackage[shortlabels]{enumitem}
\usepackage{setspace}
\usepackage{extramarks}
\usepackage{soul,color}
\usepackage{graphicx,float,wrapfig}
\newcommand{\homework}[3]{
	\pagestyle{myheadings}
	\thispagestyle{plain}
	\newpage
	\setcounter{page}{1}
	\noindent
	\begin{center}
		\framebox{
			\vbox{\vspace{2mm}
				\hbox to 6.28in { {\bf DL \hfill} {\hfill {\rm #2} {\rm #3}} }
				\vspace{4mm}
				\hbox to 6.28in { {\Large \hfill #1  \hfill} }
				\vspace{3mm}}
		}
	\end{center}
	\vspace*{4mm}
}
\newcommand\numberthis{\addtocounter{equation}{1}\tag{\theequation}}
\begin{document}
\homework{Assignment 6}{2023040165}{Zhicheng Jiang}

\subsection*{6.1}

For simplicity, we denote $u^\theta(w,h) = u$, 
and $q(w),q(\bar{w}) =q$

Then,
$$\nabla L_{NCE} = \sum_w \nabla(\tilde{p}(w|h)\log\frac{u}{u+kq}+kq\log\frac{kq}{u+kq})$$
$$ =\sum_w [\tilde{p}(w|h)\nabla\log\frac{u}{u+kq}+kq\nabla\log\frac{kq}{u+kq}]$$
$$ =\sum_w [\tilde{p}(w|h)\cdot\frac{u+kq}{u}\cdot\frac{kq\nabla u}{(u+kq)^2} - kq\frac{u+kq}{kq}\frac{-kq\nabla u}{(u+kq)^2}]$$
$$ =\sum_w [\tilde{p}(w|h)\cdot\frac{kq\nabla u}{u(u+kq)} + \frac{kq\nabla u}{u+kq}]$$
$$ =\frac{kq}{u+kq}\sum_w((\tilde{p}(w|h)-u)\frac{\nabla u}{u})$$
$$ =\frac{kq}{u+kq}\sum_w((\tilde{p}(w|h)-u)\nabla (\log u))$$
$$ \approx \sum_w((\tilde{p}(w|h)-p^\theta(w|h))\nabla (\log u))$$
$$ =\nabla L_{MLE}$$
The "$\approx$" sign achieves when$k \to \infty$, and $p^\theta(w|h) = u$

\subsection*{6.2}

\textbf{Problem 1}

1. The most computationally expensive part of 
a vannila transformer is the self-attention mechanism,
which has a time complexity of $O(n^2d)$, where $n$ is the length of the sequence and $d$ is the dimension of the input.

2. We can restrict the context window to a fixed size,
and only attend to the tokens within the window. This can reduce the time complexity to $O(nwd)$, where $w$ is the window size.

pseudo code:
\begin{verbatim}
for i in range(n):
	for j in range(max(0, i-w), min(n, i+w)):
		# calculate the attention score between i and j
\end{verbatim}

\textbf{Problem 2}

For sentiment analysis, I would suggest Alice choose BERT.
 This is because BERT is a bidirectional transformer model, 
 which can capture the context information of 
 the input sequence, and in sentiment analysis, the 
 context information is important.

Fine-tuning procedure:\\
1. Load the pre-trained BERT model.\\
2. Add a classification layer on top of the BERT model, based on the task description.\\
3. Re-train the BERT model on the sentiment analysis dataset using the classification layer.

For closed-book question answering, 
I would suggest Alice choose GPT-2. 
This is because GPT-2 is a generative transformer model,
 which can generate text based on the input sequence.
  In closed-book question answering,
   the model needs to generate the answer 
   based on the input question without accessing 
   any external knowledge.

Fine-tuning procedure:\\
1. Load the pre-trained GPT-2 model.\\
2. Re-train the GPT-2 model on the closed-book question 
answering dataset.\\




\end{document}