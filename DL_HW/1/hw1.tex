\documentclass[a4 paper,12pt]{article}
\usepackage[inner=2.0cm,outer=2.0cm,top=2.0cm,bottom=2.0cm]{geometry}
\linespread{1.1}
\usepackage{setspace}
\usepackage[rgb]{xcolor}
\usepackage{verbatim}
\usepackage{subcaption}
\usepackage{fancyhdr}
\usepackage{fullpage}
\usepackage[colorlinks=true, urlcolor=blue, linkcolor=blue, citecolor=blue]{hyperref}
\usepackage{booktabs}
\usepackage{amsmath,amsfonts,amsthm,amssymb}
\usepackage[shortlabels]{enumitem}
\usepackage{setspace}
\usepackage{extramarks}
\usepackage{soul,color}
\usepackage{graphicx,float,wrapfig}
\newcommand{\homework}[3]{
	\pagestyle{myheadings}
	\thispagestyle{plain}
	\newpage
	\setcounter{page}{1}
	\noindent
	\begin{center}
		\framebox{
			\vbox{\vspace{2mm}
				\hbox to 6.28in { {\bf DL \hfill} {\hfill {\rm #2} {\rm #3}} }
				\vspace{4mm}
				\hbox to 6.28in { {\Large \hfill #1  \hfill} }
				\vspace{3mm}}
		}
	\end{center}
	\vspace*{4mm}
}
\newcommand\numberthis{\addtocounter{equation}{1}\tag{\theequation}}
\begin{document}
\homework{Assignment 1}{2023040165}{Zhicheng Jiang}

\section*{Question 1 to 3}
False, True, False.

\section*{Question 4}

\textbf{Proof:}\\ By $\mu$-strong convexity, we have:
\begin{align*}
    f(x^*) \ge f(x^k) + \nabla f(x^k)^T(x^* - x^k) + \frac{\mu}{2}\|x^* - x^k\|^2
\end{align*}
By the definition of gradient descent and the L-Lipschitz continuity of $\nabla f$, we have:
\begin{align*}
	f(x^*) \le f(x^{k+1}) \le f(x^k) - \nabla f(x^k)^T(x^{k+1} - x^k) + \frac{1}{2L}\|\nabla f(x^k)\|^2\\
	= f(x^k) - \frac{L}{2} \|x^{k+1} - x^k\|^2 
\end{align*}\\
Combining the two inequalities, we have:
\begin{align*}
	\frac{L}{2} \|x^{k+1} - x^k\|^2 \le \nabla f(x^k)^T(x^* - x^k) + \frac{1}{2L}\|\nabla f(x^k)\|^2 + \frac{\mu}{2}\|x^* - x^k\|^2
\end{align*}
Notice that $\nabla f(x^k)=L(x^{k+1}-x^k)$, we have:
\[ \frac{L}{2} \|x^{k+1}-x^*\|^2 \le \frac{L-\mu}{2}\|x^k-x^*\|^2 \]\\
So, the number of steps required to reach $\epsilon$-accuracy is 
$O(\log_{\frac{1}{1-\frac{\mu}{L}}}\frac{R^2}{\epsilon^2})=O(\frac{L}{\mu}\log \frac{R}{\epsilon})$.

\section*{Question 5}
We have:
$\nabla f(x)=
\begin{cases}
    50x ,x < 1\\
    2x+48 ,1 \le x \le 2\\
    50x-48 , x > 2
\end{cases}$

Given $x^0=3.3$, we can easily compute that $x^1=3.3-1/9*(165-48)=-9.7$

We prove by induction that $|x^0|<|x^1|<|x^2|<\cdots<|x^{2k-1}|$, and the
even terms are positive and the odd terms are negative.(So that it won't converge)\\
The base case is true.\\
Assume that the statement is true for $k$, then we have:\\
$$x^{2k-1}<x^1<1 \to x^{2k}=x^{2k-1}-\frac{1}{9}*(50x^{2k-1})+\frac{4}{9}*(x^{2k-1}-x^{2k-2})\\=-\frac{37}{9}
x^{2k-1}-\frac{4}{9}x^{2k-2}>-x^{2k-1}$$\\
$$x^{2k}>x^0>2 \to x^{2k+1}=x^{2k}-\frac{1}{9}*(50x^{2k}-48)+\frac{4}{9}*(x^{2k}-x^{2k-1})\\$$
$$=-\frac{37}{9}x^{2k}-\frac{4}{9}x^{2k-1}+\frac{48}{9}<\frac{-35}{9}x^{2k}+\frac{48}{9}<-x^{2k}$$

Q.E.D.
\section*{Question 6}
We write
$\nabla f(x^*)=\nabla f(x^k)+\nabla^2f(x^k)(x^*-x^k)+\theta$\\
By lipchitz continuity of $\nabla^2$ , we have:
\[ \|\theta\| = O(1) \|x^*-x^k\|^2 \]
Then we know that 
\[x^{k+1}=x^k-(\nabla^2f(x^k))^{-1}(\nabla f(x^*)-\nabla^2f(x^k)(x^*-x^k)-\theta)\]
\[=x^*-\nabla^2f(x^k)^{-1}\theta\]
therefore, we have:
\[\|x^{k+1}-x^*\|=\|\nabla^2f(x^k)^{-1}\theta\| \le O(1) \|\nabla^2f(x^k)^{-1}\|\|x^*-x^k\|^2\]
By strongly convexity, $\|\nabla^2f(x^k)^{-1}\|\le \frac{1}{\mu}$, so we have:
\[\|x^{k+1}-x^*\| \le O(1) \frac{1}{\mu}\|x^k-x^*\|^2\]
Q.E.D.
\section*{Question 7}
First, we prove $\rho_Z(j)=\rho_Z(-j)$
That's because W is i.i.d Guassian, so W and -W have the same distribution.\\
Then we can know that WX and -WX have the same distribution, by linearlity, Z and -Z have the same distribution.\\
Then, after the ReLU, the variance of Z reduces by half, so we have:
\[Var(W_1x_1+W_2x_2+\cdots+W_{h_l}x_{h_l})=h_l Var(X^l)Var(W^l)\]
So, to keep it unchange, we need to have:
\[Var(W^l)=\frac{2}{h_l}\]
\end{document}