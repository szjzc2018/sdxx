\documentclass[a4 paper,12pt]{article}
\usepackage[inner=2.0cm,outer=2.0cm,top=2.0cm,bottom=2.0cm]{geometry}
\linespread{1.1}
\usepackage{setspace}
\usepackage[rgb]{xcolor}
\usepackage{verbatim}
\usepackage{subcaption}
\usepackage{fancyhdr}
\usepackage{fullpage}
\usepackage[colorlinks=true, urlcolor=blue, linkcolor=blue, citecolor=blue]{hyperref}
\usepackage{booktabs}
\usepackage{amsmath,amsfonts,amsthm,amssymb}
\usepackage[shortlabels]{enumitem}
\usepackage{setspace}
\usepackage{extramarks}
\usepackage{soul,color}
\usepackage{graphicx,float,wrapfig}
\newcommand{\homework}[3]{
	\pagestyle{myheadings}
	\thispagestyle{plain}
	\newpage
	\setcounter{page}{1}
	\noindent
	\begin{center}
		\framebox{
			\vbox{\vspace{2mm}
				\hbox to 6.28in { {\bf DL \hfill} {\hfill {\rm #2} {\rm #3}} }
				\vspace{4mm}
				\hbox to 6.28in { {\Large \hfill #1  \hfill} }
				\vspace{3mm}}
		}
	\end{center}
	\vspace*{4mm}
}
\newcommand\numberthis{\addtocounter{equation}{1}\tag{\theequation}}
\begin{document}
\homework{Assignment 3}{2023040165}{Zhicheng Jiang}
\section*{Problem 1}
False.
\section*{Problem 2}
True.
\section*{Problem 3}
For simplicity, denote $E=\mathbb{E}_{x\sim p} f(x)=\mathbb{E}_{x \sim q} \frac{p(x)f(x)}{q(x)}$
Then, we have 
\[\mathbb{V}_{x \sim q} \frac{p(x)f(x)}{q(x)} = \mathbb{E}_{x \sim q} [(\frac{p(x)f(x)}{q(x)}-E)^2]\]
\[=\mathbb{E}_{x \sim q} [(\frac{p(x)f(x)}{q(x)})^2 - 2E\cdot E + E^2]\]
\[=\mathbb{E}_{x \sim q} [(\frac{p(x)f(x)}{q(x)})^2] - E^2\]
\[=\int q(x) \frac{p(x)f(x)}{q(x)} \frac{p(x)f(x)}{q(x)} dx - E^2\]
\[=\int \frac{p^2(x)f^2(x)}{q(x)} dx \int q(x)dx - E^2\]
\[\ge (\int p(x)|f(x)|dx)^2 - E^2\]

The Cauchy-Schwarz inequality is used in the last step, and
equality holds if and only if $\frac{p(x)|f(x)|}{q(x)}$ is a constant with probability 1.
\section*{Problem 4}
1.

We check that $p(z)$ is a stationary distribution of the random walk Metropolis Hasting Algorithm:

Detailed balance:
We know that the transition probability of the random walk Metropolis Hasting Algorithm is
\[ T(s \to s')= \frac{e^{-(s'-s)^2}}{Z}min(1, \frac{p(s')}{p(s)})\]

so, we can easily check that the detailed balance holds:
\[p(s)T(s \to s') = \frac{e^{-(s'-s)^2}}{Z}p(s)min(1, \frac{p(s')}{p(s)}) \]
\[= \frac{e^{-(s'-s)^2}}{Z}min(p(s), p(s')) = \frac{e^{-(s'-s)^2}}{Z}p(s')min(1, \frac{p(s)}{p(s')}) = p(s')T(s' \to s)\]

Ergodic:
\[ \forall z,z'(p(z')>0), \]
\[ \frac{T(z\to z')}{p(z')} =\frac{e^{-(s'-s)^2}}{Z}\frac{ \min(\frac{p(z')}{p(z)},1)}{p(z')} =\frac{e^{-(s'-s)^2}}{Z}\min(\frac{1}{p(z)},\frac{1}{p(z')})\ge \frac{R}{P}\]
where $P=\max_{z} p(z), R=\frac{e^{-(Range)^2}}{Z}$

2.
It's easy to see that this is a vaild Markov chain, so it's a 
special case of the Metropolis-Hastings algorithm.

The acceptance probability is
\[ \alpha(s\to s') = \min(1, \frac{p(s')q(s'\to s)}{p(s)q(s \to s')})\]\\
However, by the definition of Gibbs sampling, we have
\[ \frac{q(s'\to s)}{q(s\to s')} = \frac{p(s)|s_{j\neq i}}{p(s')|s_{j\neq i}} = \frac{p(s)}{p(s')} \]

So, the acceptance rate is 1.

3.
WLOG suppose there're n dimensions, and a "round"
is defined as n Gibbs sampling steps among all the 
n dimensions. By the condition, we know that after 
several rounds, the transition probability is positive
for every pair of states. Let these rounds be a "Big Round".
Consider a Markov chain that uses every "Big Round"
as a step, we only need to prove that this
Markov chain has a stable distribution $p(z)$,
as we can start from any step of the "Big Round",
to get the desired distribution.

For the detailed balance part, by the equation  
in 2, we know that during each transition, the detailed balance
holds. So, it also holds for the "Big Round".

For the ergodic part, we know that the transition probability
is positive for every pair of states, 
since there're a finite number of states, thus, the ergodicity holds.

\newcommand*{\pr}[1]{\mathbb{P}{#1}}
\section*{Problem 5}
1.
The joint probability distribution is 
\[ \mathbb{P}(A,B,C,D) = \mathbb{P}(A) \mathbb{P}(B|A) \mathbb{P}(C|B) \mathbb{P}(D|A,C)\tag{*}\]
TLDR: the answer for the following questions is Dep;Ind;Dep;Dep;Dep;Dep;Dep;Ind\\
(a)\\
A and C can be dependent(e.g. A=B=C=D)\\
(b)\\
A and C are independent given B, because by (*),
\[\pr(A,C|B)=\pr(A|B)\pr(B|A,B)\pr(C|B)=\pr(A|B)\pr(C|B)\]
(c)\\
A and C can be dependent given D(e.g. A=B=C, D random)\\
(d)\\
A and C can be dependent given B,D(e.g. A,B,C are i.i.d, D=[A==C])\\
(e)\\
B and D can be dependent(e.g. A=B=C=D)\\
(f)\\
B and D can be dependent given A(e.g. B=C=D, and independent of A)\\
(g)\\
B and D can be dependent given C(e.f. A=B=D, and independent of C)\\
(h)\\
B and D are independent given A and C, because by (*),
\[\pr(B,D|A,C) = \frac{\pr(A,B,C,D)}{\pr(A,C)}\]
\[=\frac{\pr(A)\pr(B|A)\pr(C|B)\pr(D|A,C)}{\pr(A,C)}\]
\[=\frac{\pr(A,B,C)\pr(D|A,C)}{\pr(A,C)}\]
\[=\pr(B|A,C)\pr(D|A,C)\]


(2)\\
\[\pr(B,C,D|A) = \pr(B|A)\pr(C|B)\pr(D|A,C)\]
\[= \frac{\exp(\frac{-(B-A)^2-(C-B)^2-(D-C-A)^2}{2})}{\sqrt{8\pi^3}}\]

\[\pr(A|B,C,D)=\frac{e^{-[a^2+(b-a)^2+(c-b)^2+(d-a-c)^2]/2}}{\int_{a}e^{-[a^2+(b-a)^2+(c-b)^2+(d-a-c)^2]/2}}\]
\[=\frac{e^{-[\frac{3}{2}(a+(c-d-b)/3)^2]}}{\sqrt{2\pi/3}}\]

\section*{Problem 6}
1.

We induct on i*n+j to prove that the final receptive field 
of (i,j) is $R_{i,j}$=\{(i,[1:j]),(i-1,[1:j+1]),(i-2,[1:j+2]),...,(i+j-n,[1:n]),(i+j-n-1,[1:n]),...,(1,[1:n])\}.

It's easy to check that the base case holds, and $R_{i,j} = \cup_{v} R_{(i,j)-v} +(i,j)$,
where $v \in \{(0,[1:k]),([1:k],[-k:k])\}$, and denote the $R_v$ to be empty set if 
v is out of bound.

2.

We modify the mask by getting the middle line to be all zero, denote
this new mask to be $m'$. Also, we add a new "horizontal stack",
which has a mask $m''$ of all zero, except for the first half part
of the middle line(including the middle element).

The convolutional layer is represented by 
\[x'=conv(m'\cdot w, x)+z \]

And $z$ is the horizontal stack, which is intialized as $x$,
and updated by
\[z=conv(m''\cdot w,z)+z \]
\end{document}
















