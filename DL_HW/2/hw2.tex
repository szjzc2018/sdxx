\documentclass[a4 paper,12pt]{article}
\usepackage[inner=2.0cm,outer=2.0cm,top=2.0cm,bottom=2.0cm]{geometry}
\linespread{1.1}
\usepackage{setspace}
\usepackage[rgb]{xcolor}
\usepackage{verbatim}
\usepackage{subcaption}
\usepackage{fancyhdr}
\usepackage{fullpage}
\usepackage[colorlinks=true, urlcolor=blue, linkcolor=blue, citecolor=blue]{hyperref}
\usepackage{booktabs}
\usepackage{amsmath,amsfonts,amsthm,amssymb}
\usepackage[shortlabels]{enumitem}
\usepackage{setspace}
\usepackage{extramarks}
\usepackage{soul,color}
\usepackage{graphicx,float,wrapfig}
\newcommand{\homework}[3]{
	\pagestyle{myheadings}
	\thispagestyle{plain}
	\newpage
	\setcounter{page}{1}
	\noindent
	\begin{center}
		\framebox{
			\vbox{\vspace{2mm}
				\hbox to 6.28in { {\bf DL \hfill} {\hfill {\rm #2} {\rm #3}} }
				\vspace{4mm}
				\hbox to 6.28in { {\Large \hfill #1  \hfill} }
				\vspace{3mm}}
		}
	\end{center}
	\vspace*{4mm}
}
\newcommand\numberthis{\addtocounter{equation}{1}\tag{\theequation}}
\begin{document}
\homework{Assignment 2}{2023040165}{Zhicheng Jiang}
\section*{Question 1 to 2}
True. True.
\section*{Question 3}
\subsection*{1.}
To use Hopfield network to retrieve noisy images,
we can use the following steps:
\begin{enumerate}
    \item suppose the noisy image is $x \in R^{n}$
    \item each time, we update the entry $x_i$ of $x$ by
          $x_i = sign(\sum_{j=1}^{n}w_{ij}x_j)$ (i=1,2,...,n)
    \item When the update process converges, we get the
          denoised image.
\end{enumerate}
To use Hopfield network to retrieve masked images,
we can use the following steps:
\begin{enumerate}
    \item suppose the masked image is $x \in R^{n}$,
          and the masked part is $x_i (i\in I)$
    \item each time, we update the entry $x_i$ of $x$ by
          $x_i = sign(\sum_{j=1}^{n}w_{ij}x_j)$ (i=$\{1,2,...,n\}-I$)
    \item When the update process converges, we get the
          unmasked image.
\end{enumerate}
\subsection*{2.}
We can calculate that $\forall x$, 
\[
    E(x)=-\frac{1}{2N}x^T(\sum_{p}y_py_p^T)x
    =-\frac{1}{2N}\sum_{p} x^Ty_p(y_p^Tx)
    =-\frac{1}{2N}\sum_{p}||y_p^Tx||^2
    =-\frac{||x||^2}{2N}=-\frac{1}{2}
\]
Thus, all the patterns have the same energy,
that is, it memorizes $2^N$ patterns.
\section*{Question 4}
(Prove of the gradient of the Boltzmann machine)\\
\[\nabla_{W} L(W)=\nabla_{W}(-\frac{1}{|P|}\sum_{v\in P}
log(\sum_{h}\mathbb{P}(v,h)))\]
\[=\nabla_{W}(-\frac{1}{|P|}\sum_{v\in P}
log(\sum_{h}\frac{\exp(y^TWy)}{\sum_{y'}\exp(y'^TWy')}))\]
\[=\nabla_{W}(-\frac{1}{|P|}\sum_{v\in P}
log(\sum_{h}\exp(y^TWy)))+ \nabla_{W}(log{\sum_{y'}\exp(y'^TWy')})\]
\[=\nabla_{W}(-\frac{1}{|P|}\sum_{v\in P}
log(\sum_{h}\exp(y^TWy)))+ \nabla_{W}(log{\sum_{y'}\exp(y'^TWy')})\]
\[=-\frac{1}{|P|}\sum_{v\in P}
\nabla_{W}log(\sum_{h}\exp(y^TWy))+ \nabla_{W}(log{\sum_{y'}\exp(y'^TWy')})\]
\[=-\frac{1}{|P|}\sum_{v\in P}
\frac{\sum_{h}\exp(y^TWy)(y^Ty)}{\sum_{h}\exp(y^TWy)}+ \sum_y\frac{y^Ty \exp(y^TWy)}{\sum_{y'} \exp(y'^TWy')}\]
\[=-\frac{1}{|P|}\sum_{v\in P} \sum_h y^Ty \mathbb{P}(v,h)+ \sum_y y^Ty \mathbb{P}(y)\]
\[=-\frac{1}{|P|}\sum_{v\in P}(\mathbb{E}_{h|v}[yy^T]-\mathbb{E}_{y'}[y'y'^T])\]
\section*{Question 5}
(Gaussian RBM)\\
Energy:
\begin{equation}
      \mathcal{E}_{W,b}(v,h) = \frac{1}{2}(v-b)^T(v-b) - v^TWh
\end{equation}
1. 
\[ \mathbb{P}(v|h) = \frac{1}{Z_h} \exp(-\mathcal{E}_{W,b}(v,h))\]
\[ Z_h =\int_{v} \exp(-\mathcal{E}_{W,b}(v,h))dv \]
\[ =\int_{v} \exp(-\frac{1}{2}(v-b)^T(v-b) + v^TWh)dv \]
\[ =\int_{v} \exp(-\frac{1}{2}v^Tv + (v+b)^TWh)dv \]
\[ =e^{b^TWh} \int_{v} \exp(-\frac{1}{2}v^Tv + v^TWh)dv \]
\[ =e^{b^TWh+\frac{(Wh)^TWh}{2}} \int_{v} \exp(-\frac{1}{2}(v-Wh)^T(v-Wh))dv \]
\[ =e^{(b+\frac{Wh}{2})^TWh} \int_{v} \exp(-\frac{1}{2}v^2)dv\]
\[ =e^{(b+\frac{Wh}{2})^TWh} (\sqrt{2\pi})^{N_v} \]

So,
\[ \mathbb{P}(v|h) = \frac{1}{Z} \exp(-\mathcal{E}_{W,b}(v,h)) = \frac{1}{(\sqrt{2\pi})^{N_v}} \exp(-\frac{1}{2}(v-b)^T(v-b) + v^TWh - (b+\frac{Wh}{2})^TWh) \]
\[ = \frac{1}{(\sqrt{2\pi})^{N_v}} \exp(-\frac{1}{2}(v-b)^T(v-b) + v^TWh - (b^TWh+\frac{(Wh)^TWh}{2})) \]
\[ = \frac{\exp(-\frac{1}{2}(v-b)^T(v-b) + v^TWh - b^TWh-\frac{(Wh)^TWh}{2})}{(\sqrt{2\pi})^{N_v}}  \]
\[ = \frac{\exp(-\frac{1}{2}||v-b-Wh||^2)}{(\sqrt{2\pi})^{N_v}}  \]
2.
Similarly to Problem 4.2, we can get
\[ \nabla_b L(W,b) = -\frac{1}{|P|}\sum_{v\in P}(\mathbb{E}_{h|v}[\nabla_b\mathcal{E}] - \mathbb{E}[\nabla_{b}\mathcal{E}]) \]
since 
\[ \nabla_b\mathcal{E} = v-b\]
we know
\[ \nabla_b L(W,b) = -\frac{1}{|P|}\sum_{v\in P}(\mathbb{E}_{h|v}[v-b] - \mathbb{E}[v-b]) \]
\[ = - \mathbb{E}_{v \in P}[v] + \mathbb{E}_v[v] \]

\section*{Question 6}
1. D and F are not independent generally, since the joint distribution of D and F is
\[ \mathbb{P}(D|F) = \frac{\mathbb{P}(D,F)}{\mathbb{P}(F)} = \frac{\mathbb{P}(D,F)}{\sum_{D'}\mathbb{P}(D,F)} \]
\[ = \frac{f_{AD}(A,D)f_{AE}(A,E)f_{EF}(E,F)}{\sum_{D'}f_{AD}(A,D')f_{AE}(A,E)f_{EF}(E,F)}\]
which generally doesn't equal to $\mathbb{P}(D)$.

2. 
\[ \mathbb{P}(B,E|A)= \frac{\mathbb{P}(BEA)}{\mathbb{P}(A)}\]
\[ = \frac{\sum_{C'}\mathbb{P}(BEAC')}{\sum_{C'}\mathbb{P}(AC')}\]
\[ = \frac{\sum_{C'}f_{BC}(B,C')f_{AE}(A,E)f_{AC}(A,C')}{\sum_{C}f_{AC}(A,C')}\]

B and E are conditionally independent given A, because
\[ \mathbb{P}(B|A) = \frac{\mathbb{P}(BA)}{\mathbb{P}(A)} = \frac{\sum_{C'}\mathbb{P}(BAC')}{\sum_{C'}\mathbb{P}(AC')}\]
\[ = \frac{\sum_{C'}f_{BC}(B,C')f_{AC}(A,C')}{\sum_{C'}f_{AC}(A,C')}\]

\[ \mathbb{P}(E|A) = \frac{\mathbb{P}(EA)}{\mathbb{P}(A)} = \frac{\sum_{C'}\mathbb{P}(EAC')}{\sum_{C'}\mathbb{P}(AC')}\]
\[ = \frac{\sum_{C'}f_{AE}(A,E)f_{AC}(A,C')}{\sum_{C'}f_{AC}(A,C')}\]

thus, 
\[ \mathbb{P}(B|A) \mathbb{P}(E|A) \]
\[ = \frac{\sum_{C'}f_{BC}(B,C')f_{AC}(A,C')}{\sum_{C'}f_{AC}(A,C')} \frac{\sum_{C'}f_{AE}(A,E)f_{AC}(A,C')}{\sum_{C'}f_{AC}(A,C')} \]
\[ = \frac{\sum_{C'}f_{BC}(B,C')f_{AC}(A,C')f_{AE}(A,E)f_{AC}(A,C')}{(\sum_{C'}f_{AC}(A,C'))^2} \]
\[ = \frac{\sum_{C'}f_{BC}(B,C')f_{AE}(A,E)f_{AC}(A,C')}{\sum_{C'}f_{AC}(A,C')} \]
\[ = \frac{\sum_{C'}f_{BC}(B,C')f_{AE}(A,E)f_{AC}(A,C')}{\sum_{C}f_{AC}(A,C')} \]
\[ = \mathbb{P}(B,E|A) \]

3.
That's because the joint probability
\[ \mathbb{P}(A,B,C,D,E,F) = \frac{1}{Z} f_{BC}(B,C)f_{AC}(A,C)f_{AD}(A,D)f_{AE}(A,E)f_{EF}(E,F) \]
So the energy function $\mathcal{E}(A,B,C,D,E,F)$ can be expressed
by $\mathcal{E}(A,C), \mathcal{E}(A,D), \mathcal{E}(A,E),$\\$ \mathcal{E}(B,C), \mathcal{E}(E,F)$ 
(just let the energy function be the minus log of the joint probability)
\end{document}